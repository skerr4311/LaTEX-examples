\documentclass[a4paper,12pt]{article}
\usepackage{mathtools}
\usepackage{tikz}
\usepackage{enumitem}
\usetikzlibrary{automata, positioning}
\usepackage{pgf}
\usepackage{wrapfig,lipsum,booktabs}
\usepackage{mathptmx,amssymb,amsmath}
\usetikzlibrary{arrows,automata}
\usepackage[latin1]{inputenc}
\usepackage{verbatim}
\usepackage{listings}
\usepackage{color}

\definecolor{dkgreen}{rgb}{0,0.6,0}
\definecolor{gray}{rgb}{0.5,0.5,0.5}
\definecolor{mauve}{rgb}{0.58,0,0.82}

\lstset{frame=tb,
  language=Java,
  aboveskip=3mm,
  belowskip=3mm,
  showstringspaces=false,
  columns=flexible,
  basicstyle={\small\ttfamily},
  numbers=none,
  numberstyle=\tiny\color{gray},
  keywordstyle=\color{blue},
  commentstyle=\color{dkgreen},
  stringstyle=\color{mauve},
  breaklines=true,
  breakatwhitespace=true,
  tabsize=3
}

\title{Computer Science 220 \\
\large Assignment One}
\author{Steven Kerr 6022796}
\date{01/08/2019}

\begin{document}
\maketitle

\noindent \textbf{Q1. (a)}
\begin{equation}
\begin{split}
T_1(n)={}& 10^4n \\
		={}& 10 \cdot 10 \cdot 10 \cdot 10 \cdot n \\
		={}& 10,000 \cdot n \\
T_2(n)={}& 10^n \\
T_3(n)={}& 10^3 \cdot n^3 \cdot  \log_{10} n\\
		={}& (10 \cdot 10 \cdot 10) \cdot n^3 \cdot \log_{10} n\\
		={}& 1,000 \cdot n^3 \cdot \log_{10} n \\
\end{split}
\end{equation}
if $n=10^{10}$

\begin{equation}
\begin{split}
T_1(10^{10})={}& 10^4n \\
		={}& 10,000 \cdot 10^{10} \\
		={}& 10,000 \cdot 10,000,000,000 \\
		={}& 100,000,000,000,000 \\
T_2(10^{10})={}& 10^{10,000,000,000} \\
		={}& 10 \cdot 10 \cdot 10 \cdot \dotsc \cdot 10 (n\; times)\\
		={}& undefined \\
T_3(10^{10})={}& 1,000 \cdot ((10^{10})^3 \cdot 10\\
		={}& 1,000 \cdot (10,000,000,000)^3 \cdot 10\\
		={}& undefined\\
\end{split}
\end{equation}
\newpage
As we can see above $10^{10,000,000,000}$ is much larger than $1,000 \cdot (10,000000000)^3 \cdot 10$. I will demonstrate below:
$$(10^{10})^{10} \; > \; 10^3 \cdot (10^{10})^3 \cdot 10 \; > \; 10^4 \cdot 10^{10}$$
$$10^{100} \; > \; 10^4 \cdot 10^{30} \; > \; 10^{14}$$
$$10^{100} \; > \; 10^{34} \; > \; 10^{14}$$
$$T_2 \; > \; T_3 \; > \; T_1$$
Therefore $T_1$ will be the fastest and $T_2$ will be the slowest.\\
\\
\noindent \textbf{Q1. (b)}\\
if $n=10$: $T_1$ is the best algorithm
$$ T_2 \; > \; T_3 \; > \; T_1$$
$$ 10^{n} \; > \; 10^3 \cdot n^3 \cdot  \log_{10} n \; > \; 10^4n $$
$$ 10^{10} \; > \; 10^3 \cdot 10^3 \cdot  \log_{10} 10 \; > \; 10^4 \cdot 10 $$
$$ 10^{10} \; > \; 10^6 \cdot 1 \; > \; 10^5 $$
$$ 10^{10} \; > \; 10^6 \; > \; 10^5 $$

if $n=3$: $T_2$ is the best algorithm
$$ T_1 \; > \; T_3 \; > \; T_2$$
$$ 10^4n \; > \; 10^3 \cdot n^3 \cdot  \log_{10} n \; > \; 10^{n} $$
$$ 10^4 \cdot 3 \; > \; 10^3 \cdot 3^3 \cdot  \log_{10} 3 \; > \; 10^{3} $$
$$ 10,000 \cdot 3 \; > \; 1,000 \cdot 27 \cdot  \log_{10} 3 \; > \; 1,000 $$
$$ 30,000 \; > \; 27,000 \cdot  \log_{10} 3 \; > \; 1,000 $$
$$ 30,000 \; > \; 12,882 \; > \; 1,000 $$

if $n=1$: $T_3$ is the best algorithm 
$$ T_1 \; > \; T_2 \; > \; T_3$$
$$ 10^4n \; > \; 10^{n} \; > \; 10^3 \cdot n^3 \cdot  \log_{10} n$$
$$ 10^4 \cdot 1 \; > \; 10^{1} \; > \; 10^3 \cdot 1^3 \cdot  \log_{10} 1$$
$$ 10,000 \; > \; 10 \; > \; 1000 \cdot 1 \cdot  0$$
$$ 10,000 \; > \; 10 \; > \; 0$$

\newpage
\noindent \textbf{Q2. (a)}\\
The first thing i will do is try to simplify the question.
\begin{equation}
\begin{split}
\lim\limits_{x \to \infty}={}& \frac{2^n + 2^{n+1}}{2^n - 2^{n-1}} \\
							\\
							={}& \frac{2^n + 2^1 \cdot 2^n}{2^n - 2^{-1} \cdot 2^n} \\
							\\
							={}& \frac{2^n + 2 \cdot 2^n}{2^n \cdot (-\frac{1}{2} + 1)} \\
							\\
							={}& \frac{2^n \cdot (1 + 2)}{2^n \cdot (1-\frac{1}{2})} \\
							\\
							={}& \frac{3}{(1-\frac{1}{2})} \\
							\\
							={}& \frac{3}{\frac{1}{2}} \\
							\\
							={}& \frac{6}{1} \\
\end{split}
\end{equation}
As shown above, no matter what the value of $n$ the above function will always output $6$. Therefore
$$\lim\limits_{x \to \infty} = \frac{2^n + 2^{n+1}}{2^n - 2^{n-1}} = 6$$

\newpage

\noindent \textbf{Q2. b()}\\
Firstly, applying the log rule $\log_c(\frac{a}{b}) \; = \; log_c(a)-\log_c(b)$ we then have 
$$\log_{2}(\frac{1}{\log_2(n)})\;=\; \log_2(1)-\log_2(\log_2(n))$$
We know that $\log_2(1)=0$ and so we have: $$\log_{2}(\frac{1}{\log_2(n)})\;=\; 0-\log_2(\log_2(n))$$
for all values of $n$ that are plugged into $\log_2(n)$ we get some positive number returned. We also know from class that: $$\lim\limits_{x \to \infty}\log_2(n) = \infty$$
we can also deduce from this information that:
$$\lim\limits_{x \to \infty}\log_2(\log_2(n)) = \infty$$
Because we have $0-\log_2(\log_2(n))$ this means that we will get back a negative number. Therefore: $$\lim\limits_{x \to \infty}\log_2(\frac{1}{\log_2(n)}) \; = \ - \infty$$

\noindent \textbf{Q2. (c)}\\
I first tried adding values of $n$ to the equation to see what came out. I noticed the equation rapidly started heading for $1$ as the size of inputs got larger and larger. To show this I tried to simplify the equation. I applied the cube of sum rule : 
$$a^3+3a^2b+3ab^2+b^3=(n+1)^3$$ which gave me $$n^3+3n^21+3n1^2+1^3=(n+1)^3$$ if i apply the same technique to the denominator then I get:
$$n^3-3n^21+3n1^2-1^3=(n-1)^3$$
Now plugging these values back in:
$$\frac{n^3+3n^2+3n^2+1}{n^3-3n^2+3n^2-1} = \frac{(n+1)^3}{(n-1)^3}$$
Based on observation as $n$ heads to infinity, the remainder of the division gets smaller and smaller. Because we can see a difference of 2 numbers between the numerator and the denominator i.e: $n=10$ then $\frac{(n+1)^3}{(n-1)^3}=\frac{11}{9}$\\
We know that: $\frac{(n+1)^3}{(n-1)^3}=$ 1 + (some remainder $k$). Based on this observation I conclude that $$\lim\limits_{x \to \infty}\frac{n^3+3n^2+3n^2+1}{n^3-3n^2+3n^2-1} = 1$$
\noindent \textbf{Q3. }\\
For this question the worst case senario would be for all while loops to be hit. I start by working on the center most one of the below code:
\begin{lstlisting}
//Question 3
if n <= 1 {
	return 0
}else{
	i = 0
	j = 1
	k = 1
	while i <= n {
		while j <= n {
			while k <= n {
				constant number C of elemental operations
				k = k * n
			}
			j = j * 3
		}
		i = i + 5
	}
return some number
\end{lstlisting}
\begin{lstlisting}
//while loop 1
while k <= n {		//one operation to compare
	constant number C of elemental operations //One constant C
	k = k * n		//two operations. one for k*n and one for k = k*n
}
\end{lstlisting}
For while loop 1, It will only run through twice. This is because no matter the value of $n$, $k=(k\cdot n)\;=\; n$. This is because $k=1$ and $n\cdot1=n$. The loop runs twice, once when $k=1$, once when $k=n$ and one last time without running the operations in the loop. This means that the run time of loop 1 is:
$$2(C+2)+1$$
\newpage
\begin{lstlisting}
//while loop 2
while j <= n {		//one operation to compare
	//while loop 1 = 2(C+2)+1
	j = j * 3		//two operations. one for k*n and one for k = k*n
}
\end{lstlisting}
For while loop 2 i started working out what $j$ will do i,e:
$$j = 1, 3, 9, 27, 81 \dots$$
$$j=3^0, 3^1, 3^2, 3^3, 3^4 \dots$$
Because i know from class that:
$$3^m = n \iff \log_3(n)=m$$
i know that the while loop 2 will execute $\log_3(n)+1$ times. This is because when $j$ starts, $j=1$ and $\log_3(n) = 0$ so to adjust for the first loop being made we add $1$. Next, the operations in the loop are equal to $2$ and finally we again compensate for the case where $k=n$ by adding another $1$. therefore, while loop 2 has a run time of:
$$2\log_3(n)+1+1$$ 
\begin{lstlisting}
//while loop 3
while i <= n {		//one operation to compare
	//while loop 1 = 2(C+2)+1
	//while loop 2 = 2log_3(n)+1+1
	i = i +5		//two operations. one for k*n and one for k = k*n
}
\end{lstlisting}
Again, for while loop 3 I check the values of $i$;
$$i = 5, 10, 15, 20, 25, 30, \dots$$
which looks like we have a runtime of $\frac{n}{5}$. Because we know that we need to add $1$ to compensate for when $i=n$ we add another $1$. We also have another two operations in the while loop. therefore, while loop 3 has a run time of:
$$2\cdot\frac{n}{5}+1$$ 
For all of the loops we have and adding the initial 4 operations of the if statement check and assigning values to $i, j, k$ the total runtime "worst case scenario would be":
$$4+2(C+2)+1+2\log_3(n)+1+1+2\cdot\frac{n}{5}+1$$
$$=2C+\frac{2n}{5}+\log_3(n)+12$$

\newpage
\noindent \textbf{Q4.}\\
\begin{lstlisting}
//question 4
if n < 1 {
	i = 2
	while i <= n{
		constant number C
		i = i*i
	}
}else{
	for i=1 to n {
		constant number C
	}
}
return some number
\end{lstlisting}
There are two cases here. If $n < 1$ and if $n >= 1$.\\ 
\\
If $n < 1$ Then $i=2$. That is two operations so far. The while loop checks to see if $i <= n$, that is another operation. since we know that $n < 1$ we know that $n$ can never be greater than $i$ because $i=2$ so the loop never runs. the return statement is hit and we finish with $4\; operations$.\\
\\
Next, if $n >= 1$.\\
let $n=1$, we check $n < 1$ and that is our first operation. We move to the 'else' statement. In the 'for loop', since $n=1$ and $i=1$ we never enter the for loop and hit return. This is 3 operations and the best case scenario.\\
\\
There is a 3rd case where $n=$'some large number'.\\
In this case we would check the first 'if' statement, move to the else statement, move to the 'for' loop and execute that loop $(n-1)$ times each time executing 'C', then finally to the return statement. This would be the worst case scenario at $2+ (n-1)\cdot C$
\end{document}
