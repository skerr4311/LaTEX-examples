\documentclass[a4paper,12pt]{article}
\usepackage{mathtools}
\usepackage{tikz}
\usepackage{enumitem}
\usetikzlibrary{automata, positioning}
\usepackage{pgf}
\usepackage{wrapfig,lipsum,booktabs}
\usepackage{mathptmx,amssymb,amsmath}
\usetikzlibrary{arrows,automata}
\usepackage[latin1]{inputenc}
\usepackage{verbatim}

\title{Computer Science 220 \\
\large Assignment One}
\author{Steven Kerr 6022796}
\date{01/08/2019}

\begin{document}
\maketitle

\noindent \textbf{Q1. (a)}
\begin{equation}
\begin{split}
T_1(n)={}& 10^4n \\
		={}& 10 \cdot 10 \cdot 10 \cdot 10 \cdot n \\
		={}& 10,000 \cdot n \\
T_2(n)={}& 10^n \\
T_3(n)={}& 10^3 \cdot n^3 \cdot  \log_{10} n\\
		={}& (10 \cdot 10 \cdot 10) \cdot n^3 \cdot \log_{10} n\\
		={}& 1,000 \cdot n^3 \cdot \log_{10} n \\
\end{split}
\end{equation}
if $n=10^{10}$

\begin{equation}
\begin{split}
T_1(10^{10})={}& 10^4n \\
		={}& 10,000 \cdot 10^{10} \\
		={}& 10,000 \cdot 10,000,000,000 \\
		={}& 100,000,000,000,000 \\
T_2(10^{10})={}& 10^{10,000,000,000} \\
		={}& 10 \cdot 10 \cdot 10 \cdot \dotsc \cdot 10 (n\; times)\\
		={}& undefined \\
T_3(10^{10})={}& 1,000 \cdot ((10^{10})^3 \cdot 10\\
		={}& 1,000 \cdot (10,000,000,000)^3 \cdot 10\\
		={}& undefined\\
\end{split}
\end{equation}
\newpage
As we can see above $10^{10,000,000,000}$ is much larger than $1,000 \cdot (10,000000000)^3 \cdot 10$. I will demonstrate below:
$$(10^{10})^{10} \; > \; 10^3 \cdot (10^{10})^3 \cdot 10 \; > \; 10^4 \cdot 10^{10}$$
$$10^{100} \; > \; 10^4 \cdot 10^{30} \; > \; 10^{14}$$
$$10^{100} \; > \; 10^{34} \; > \; 10^{14}$$
$$T_2 \; > \; T_3 \; > \; T_1$$
Therefore $T_1$ will be the fastest and $T_2$ will be the slowest.\\
\\
\noindent \textbf{Q1. (b)}\\
if $n=10$: $T_1$ is the best algorithm
$$ T_2 \; > \; T_3 \; > \; T_1$$
$$ 10^{n} \; > \; 10^3 \cdot n^3 \cdot  \log_{10} n \; > \; 10^4n $$
$$ 10^{10} \; > \; 10^3 \cdot 10^3 \cdot  \log_{10} 10 \; > \; 10^4 \cdot 10 $$
$$ 10^{10} \; > \; 10^6 \cdot 1 \; > \; 10^5 $$
$$ 10^{10} \; > \; 10^6 \; > \; 10^5 $$

if $n=3$: $T_2$ is the best algorithm
$$ T_1 \; > \; T_3 \; > \; T_2$$
$$ 10^4n \; > \; 10^3 \cdot n^3 \cdot  \log_{10} n \; > \; 10^{n} $$
$$ 10^4 \cdot 3 \; > \; 10^3 \cdot 3^3 \cdot  \log_{10} 3 \; > \; 10^{3} $$
$$ 10,000 \cdot 3 \; > \; 1,000 \cdot 27 \cdot  \log_{10} 3 \; > \; 1,000 $$
$$ 30,000 \; > \; 27,000 \cdot  \log_{10} 3 \; > \; 1,000 $$
$$ 30,000 \; > \; 12,882 \; > \; 1,000 $$

if $n=1$: $T_3$ is the best algorithm 
$$ T_1 \; > \; T_2 \; > \; T_3$$
$$ 10^4n \; > \; 10^{n} \; > \; 10^3 \cdot n^3 \cdot  \log_{10} n$$
$$ 10^4 \cdot 1 \; > \; 10^{1} \; > \; 10^3 \cdot 1^3 \cdot  \log_{10} 1$$
$$ 10,000 \; > \; 10 \; > \; 1000 \cdot 1 \cdot  0$$
$$ 10,000 \; > \; 10 \; > \; 0$$

\newpage
\noindent \textbf{Q2. (a)}\\
The first thing i will do is try to simplify the question.
\begin{equation}
\begin{split}
\lim\limits_{x \to \infty}={}& \frac{2^n + 2^{n+1}}{2^n - 2^{n-1}} \\
							\\
							={}& \frac{2^n + 2^1 \cdot 2^n}{2^n - 2^{-1} \cdot 2^n} \\
							\\
							={}& \frac{2^n + 2 \cdot 2^n}{2^n \cdot (-\frac{1}{2} + 1)} \\
							\\
							={}& \frac{2^n \cdot (1 + 2)}{2^n \cdot (1-\frac{1}{2})} \\
							\\
							={}& \frac{3}{(1-\frac{1}{2})} \\
							\\
							={}& \frac{3}{\frac{1}{2}} \\
							\\
							={}& \frac{6}{1} \\
\end{split}
\end{equation}
As shown above, no matter what the value of $n$ the above function will always output $6$. Therefore
$$\lim\limits_{x \to \infty} = \frac{2^n + 2^{n+1}}{2^n - 2^{n-1}} = \infty$$
\end{document}
